\documentclass{llncs}

%\usepackage{makeidx}  % allows for indexgeneration

\usepackage{mathrsfs} % for mathscr
\usepackage{algorithmic}
\usepackage{algorithm}

\usepackage{amsmath}
\usepackage{amssymb}

\usepackage{color}
\usepackage{subfigure}
\usepackage[table]{xcolor}
\usepackage{graphicx}
\usepackage{hyperref}

\hypersetup{colorlinks,linkcolor={red!50!black},citecolor={blue!50!black},urlcolor={blue!80!black}}

\newcommand{\f}[1]{\mathtt{#1}} %\tt #1} % field
\newcommand{\h}[1]{\textit{#1}} % heuristic
\newcommand{\p}[1]{\textit{#1}} % problem

%\ifx\pdftexversion\undefined
%  \usepackage[dvips]{graphics}
%\else
%  \usepackage[pdftex]{graphics}
%\fi

%\newcommand{\sub}[1]{\xrightarrow[ _{#1}]{}}

\begin{document}

%\pagestyle{headings}  % switches on printing of running heads
    \pagestyle{empty}


    \title{CoSoCo 2.5\\ {\small XCSP3 Competition 2025}}

    \author{Gilles Audemard}

    \institute{CRIL-CNRS, UMR 8188\\
    Universit\'e d'Artois, F-62307 Lens France\\
    \email{audemard@cril.fr}\\
    }
%\date{03 September 2017}

%\begin{document}

    \maketitle




    CoSoCo is a constraint solver written in
    C++. The main goal is to build
    a simple, but efficient constraint solver. CoSoCo is available on
    \href{https://github.com/xcsp3team/cosoco}{github}. CoSoCo recognizes XCSP3
    \cite{BLPPxcsp3} by using the C++ parser that can be downloaded at
    \href{https://github.com/xcsp3team/XCSP3-CPP-Parser}{https://github.com/xcsp3team/XCSP3-CPP-Parser}. It
    can deal with all XCSP3 Core constraints.
    This is the seventh participation of CoSoCo to XCSP competitions.

    \bigskip
    CoSoCo performs backtrack search, enforcing (generalized) arc consistency at each node (when possible).
    The main components are :
    \begin{itemize}
        \item \h{lexico} as value ordering heuristic;
        \item lc(1), last-conflict reasoning with a collecting parameter $k$ set to $1$, as described in \cite{LSTV_reasonning};
        \item a variable-oriented propagation scheme \cite{G_relational}, where a queue $Q$ records all variables with recently reduced domains (see Chapter 4 in \cite{L_constraint}).
        \item The solution saving technique \cite{vion2017}.
    \end{itemize}



    This year new additional features are embedded:
    \begin{itemize}
        \item A complete AllDif propagator \cite{regin94}.
        \item A better decomposition of intensional constraint.
        \item Additionnal table constraint propagator.
        \item A parallel engine using pFactory (\url{https://github.com/crillab/pfactory}).
    \end{itemize}



    \medskip
    This year CoSoCo comes with 2 modes per track. The different options used are defined below (in bold, you can see which version obtained the best results).

    \begin{itemize}
        \item     CSP
        \begin{itemize}
            \item   {\bf v1}: {\tt -var=wdeg -restarts=io -stick=1}
            \item v2: {\tt -var=wdeg -val=robin -restarts=io}
        \end{itemize}
        \item COP/Fast COP
        \begin{itemize}
            \item {\bf v1}: {\tt -var=robin -restarts=io -val=max}
            \item v2: {\tt -var=robin -val=robin -restarts=io}
        \end{itemize}
        \item Parallel COP
        \begin{itemize}
            \item {\bf v1}: {\tt  -mem\_lim=MEMLIMIT -nbcores=8}
        \end{itemize}
    \end{itemize}





    \section*{Acknowledgements}
    This work would not have taken place without Christophe Lecoutre. I would like to thank him very warmly for his support.

    \bibliographystyle{plain}
    \bibliography{globalBiblio}


\end{document}

